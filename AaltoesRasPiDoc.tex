\documentclass{article}
\usepackage[utf8]{inputenc}
\setlength{\parskip}{1em}

\usepackage{fancyhdr}
\usepackage{amsmath}

\fancyhf{}
\usepackage{listings}

\usepackage{graphicx}
\usepackage{color}
\sloppy
\definecolor{lightgray}{gray}{0.5}
\setlength{\parindent}{0pt}
\graphicspath{ {images/} }

\usepackage{amsfonts,amssymb,amsbsy}
\usepackage{hyperref}
\hypersetup{pdfpagemode=UseNone, pdfstartview=FitH,
  colorlinks=true,urlcolor=red,linkcolor=blue,citecolor=black,
  pdftitle={Default Title, Modify},pdfauthor={Your Name},
  pdfkeywords={Modify keywords}}

\title{Infoscreen RaspberryPi documentation}
\author{Tuukka Rouhiainen \& Sauli Sjögren}

\usepackage{cite}

\begin{document}

\date{13.12.2016}
\maketitle
\thispagestyle{fancy}

\section{Description}
This is short documentation of RaspberryPi`s settings and installed packages for Infoscreen which is running in Startup Sauna.
\section{IP addresses}
These IP's aren't static so they doesn't necessarily hold.

Left: \textbf{10.100.14.75} \\
Right: \textbf{10.100.28.220}
\section{Installed packages}
\textbf{cec-utils} - This package is to control CEC (Consumer Electronic Control) supported TV via RaspberryPi.

\textbf{unclutter} - Used to remove mouse cursor.

\textbf{xautomation}

\textbf{iceweasel}
\section{Config files}
\subsection{Crontab}
To edit crontab, write \textbf{crontab -e}.

\lstinputlisting[language=bash,firstline=24]{scripts/crontab}

\subsection{Autostart}
In NOOBS autostart file is located at $\sim/.config/lxsession/LXDE-pi$ \\
In Rasbian Pixel file is located at $\sim/.config/lxsession/LXDE/autostart$

\lstinputlisting[language=bash]{scripts/autostart}

\subsection{Fullscreen script}
\lstinputlisting[language=bash]{scripts/fullscreen.sh}
\subsection{Script to toggle TV on/off}
\lstinputlisting[language=bash]{scripts/toggleMonitor.sh}
\section{Facebook events}
Python script requires tokens given file facetoken.conf. Format is:
\begin{verbatim}
[FACEBOOK]
app_id:XXX
app_secret:XXX
\end{verbatim}
Current fetched Facebook events are from: \textbf{aaltoes, sosaaltoes, startupsauna}
You can change them in getEventsV4.py

\end{document}

