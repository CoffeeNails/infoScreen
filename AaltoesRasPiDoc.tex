\documentclass{article}
\usepackage[utf8]{inputenc}
\setlength{\parskip}{1em}

\usepackage{fancyhdr}
\usepackage{amsmath}

\fancyhf{}


\usepackage{graphicx}
\usepackage{color}
\sloppy
\definecolor{lightgray}{gray}{0.5}
\setlength{\parindent}{0pt}
\graphicspath{ {images/} }

\usepackage{amsfonts,amssymb,amsbsy}
\usepackage{hyperref}
\hypersetup{pdfpagemode=UseNone, pdfstartview=FitH,
  colorlinks=true,urlcolor=red,linkcolor=blue,citecolor=black,
  pdftitle={Default Title, Modify},pdfauthor={Your Name},
  pdfkeywords={Modify keywords}}

\title{Infoscreen RaspberryPi documentation}
\author{Tuukka Rouhiainen \& Sauli Sjögren}

\usepackage{cite}

\begin{document}

\date{13.12.2016}
\maketitle
\thispagestyle{fancy}

\section{Description}
This is short documentation of RaspberryPi`s settings and installed packages for Infoscreen which is running in Startup Sauna.
\section{IP addresses}
These IP's aren't static so they doesn't necessarily hold.

Left: \textbf{10.100.45.65} \\
Right: \textbf{10.100.51.44}
\section{Installed packages}
\textbf{cec-utils} - This package is to control CEC (Consumer Electronic Control) supported TV via RaspberryPi.

\textbf{unclutter} - Used to remove mouse cursor.

\textbf{xautomation}

\textbf{iceweasel}
\section{Config files}
\subsection{Crontab}
To edit crontab, write \textbf{crontab -e}.
\begin{verbatim}
0 2 * * * /home/pi/toggleMonitor.sh off >/dev/null 2>&1
0 6 * * * /home/pi/toggleMonitor.sh on >/dev/null 2>&1

8 /6 * * * sudo rm -r ~/.cache/mozilla/firefox/*.default/* >/dev/null 2>&1
10 /6 * * * /home/pi/public_html/BusScreen/javascript/getEventsV4.py >/dev/null 2>&1
13 /6 * * * mv /home/pi/events.json /home/pi/public_html/BusScreen/javascript/events.json >/dev/null 2>&1
\end{verbatim}
\subsection{Autostart}
In NOOBS autostart file is located at $\sim/.config/lxsession/LXDE-pi$ \\
In Rasbian Pixel file is located at $\sim/.config/lxsession/LXDE/autostart$
\begin{verbatim}
@xscreensaver -no-splash
@xset s off
@xset -dpms
@sudo xset s noblank
@unclutter
@/home/pi/fullscreen.sh
\end{verbatim}
\subsection{Fullscreen script}
\begin{verbatim}
sudo rm -r ~/.cache/mozilla/firefox/*.default/*
iceweasel localhost/~pi/BusScreen/index.html --display=:0 &
sleep 10;
xte  "key F11" -x:0
\end{verbatim}
\subsection{Script to toggle TV on/off}
\begin{verbatim}
#!/bin/bash -e

CMD="$1"

function main {
    if [ "$CMD" == "on" ]; then
	echo "on 0" | cec-client -s 
    elif [ "$CMD" == "off" ]; then
	echo "standby 0" | cec-client -s
    fi
    exit 0
}

main
\end{verbatim}
\section{Facebook events}
Python script requires tokens given file facetoken.conf. Format is:
\begin{verbatim}
[FACEBOOK]
app_id:XXX
app_secret:XXX
\end{verbatim}
Current fetched Facebook events are from: \rextbf{aaltoes, sosaaltoes, startupsauna}
You can change them in getEventsV4.py

\end{document}

